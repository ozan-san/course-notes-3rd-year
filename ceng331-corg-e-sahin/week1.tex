\documentclass{article}
\usepackage[utf8]{inputenc}
\usepackage{amssymb}
\usepackage{amsmath}
\usepackage{color}
\usepackage{listings}
\definecolor{mygreen}{rgb}{0,0.3,0}
\definecolor{mygray}{rgb}{0.3,0.3,0.3}
\definecolor{mymauve}{rgb}{0.58,0,0.82}
\newcommand{\ms}[1]{\texttt{#1}} 
\title{CENG331 - Computer Organization Course Notes}
\author{Ozan Şan}
\date{\today}
\lstset{ %
  backgroundcolor=\color{white},   % choose the background color
  basicstyle=\footnotesize,        % size of fonts used for the code
  breaklines=true,                 % automatic line breaking only at whitespace
  captionpos=b,                    % sets the caption-position to bottom
  commentstyle=\color{mygreen},    % comment style
  escapeinside={\%*}{*)},          % if you want to add LaTeX within your code
  keywordstyle=\color{blue},       % keyword style
  stringstyle=\color{mymauve},     % string literal style
}
\begin{document}

\maketitle

\section{Bits}

A common misconception: \\
$int = \mathbb{Z}$ and, $float = \mathbb{R}$ .\\

For the case of \textit{int}, $x^{2} < 0$ can happen. \\

For the case of \textit{float}s, $(x+y)+z \neq x+(y+z)$ is possible. Commutativity is not guaranteed.

\subsection{How is a C code compiled?}

\subsubsection{Preprocessor}
\begin{table}[ht!]
\centering
\begin{tabular}{lll}
\cline{2-2}
\multicolumn{1}{l|}{$\rightarrow$} & \multicolumn{1}{l|}{Preprocessor (cpp)} & $\rightarrow$       \\ \cline{2-2}
\texttt{hello.c}                &                        & \texttt{hello.i}
\end{tabular}
\end{table}

In this stage, \texttt{\#include} and \texttt{\#define} statements are processed. This happens, for example, for \texttt{\#include}, a file is copied to the beginning of \texttt{hello.c} and \texttt{\#define} macros are replaced in-place.


\subsubsection{Compiler}
\begin{table}[ht!]
\centering
\begin{tabular}{lll}
\cline{2-2}
\multicolumn{1}{l|}{$\rightarrow$} & \multicolumn{1}{l|}{Compiler (cc1)} & $\rightarrow$       \\ \cline{2-2}
\texttt{hello.i}               &                        & \texttt{hello.s (ASM)}
\end{tabular}
\end{table}

In this stage, the c code without preprocessor stages passes through a compiler and results in an Assembly code block. This does not have the necessary functions in it (for example, \texttt{printf}).
\vspace{10px}

\subsubsection{Assembler}
Here is the assembler in action: \\
\begin{center}
\begin{table}[ht!]
\centering
\begin{tabular}{lll}
\cline{2-2}
\multicolumn{1}{l|}{$\rightarrow$} & \multicolumn{1}{l|}{Assembler (as)} & $\rightarrow$       \\ \cline{2-2}
\texttt{hello.s}               &                        & \texttt{hello.o}
\end{tabular}
\end{table}
\end{center}
In this stage, Assembler takes in the Assembly code, and produces an almost-executable intermediate \textbf{Object file}. This \textbf{Object file} does not have any externally defined functions in it yet.

\subsubsection{Linker}
Here is the linker in action: \\
\begin{center}
\begin{table}[ht!]
\centering
\begin{tabular}{lll}
\texttt{printf.o} $\searrow$                        &                             &                             \\ \cline{2-2}
\multicolumn{1}{l|}{\texttt{hello.o} $\rightarrow$} & \multicolumn{1}{l|}{Linker} &$\rightarrow$ \texttt{hello} (executable) \\ \cline{2-2}
\end{tabular}
\end{table}

At the end, we have an executable in our hands. \textit{Yay.}
\end{center}

\subsection{Bit Operations}
\subsubsection{Representing sets as bits}

Take, for example, the set $S_1 = \{0,3,5,6\}$. We can cram this set into a byte, as this: \texttt{01101001}. Each bit $X_i$ is set to $1$ if $i \in S_1$. As another example, let's take $S_2 = \{0,2,4,6\}$ and its corresponding bit representation, \texttt{01010101}. \\

Now, we can use mathematical operations on these bit representations provided by the processor:

\begin{center}
\begin{table}[ht!]
\centering
\begin{tabular}{|l|l|l|l|}
\hline
Bit Operation & Mathematical Operation & Bit Result & Set result \\ \hline
\texttt{$\&$}                 & \ms{Intersection} ($\bigcap$)                                    & \ms{01000001}              & $\{0, 6\}$                    \\ \hline
\texttt{$|$}                 & \ms{Union} ($\bigcup$)                                    & \ms{01111101}              & $\{0,2,3,4,5,6\}$                   \\ \hline
\texttt{$\wedge$}                 & \ms{Symmetric Difference}                                   & \ms{00111100}             & $\{2,3,4,5\}$                   \\ \hline
\texttt{$\thicksim$}                & \ms{Complement}                                   & \ms{10010110} (On $S_1$)             & $\{1,2,4,7\}$                   \\ \hline
\end{tabular}
\end{table}
\end{center}

An important point to make is that these bitwise operations are not the same as logical operators (for example, \ms{\&\&}) as logical operators operate on the principle that everything non-zero can be treated as \ms{True}, and this means we will not get a result beyond the first bit (as the result will only be either 0, or 1). Also, logical operators can terminate early for making computations faster. (\ms{!!0x41 $\neq$ 0x41}, but, \ms{!!0x41 = 0x01 (True)}) \\
An example usage of the bitwise operations is below. We will try to write a \ms{void swap(int* a, int* b)} function without an additional variable. We can achieve this by using XOR.

\begin{center}
\centering
\begin{lstlisting}[language=C]
void swap(int* a, int* b)
{
	*a = *a ^ *b; //1
	*b = *a ^ *b; //2
	*a = *a ^ *b; //3
}
\end{lstlisting}
\end{center}
Here's what happens when we execute this, line by line:\\
\begin{center}
\begin{table}[ht!]
\centering
\begin{tabular}{|l|l|l|}
\hline
Line & Value of A                     & Value of B                     \\ \hline
1    & \ms{$A \oplus B$}              & \ms{$B$}                       \\ \hline
2    & \ms{$A \oplus B$}              & \ms{$B \oplus B \oplus A = A$} \\ \hline
3    & \ms{$A \oplus B \oplus A = B$} & \ms{$A$}                       \\ \hline
End  & \ms{$B$}                         & \ms{$A$}                         \\ \hline
\end{tabular}
\end{table}
\end{center}
With bit operations, we have saved on some storage. \textit{Yay.}
\subsubsection{Shifting}
We can shift the bits in a variable. Left shift operations replace the missing bits by 0's, whereas right shift operations may or may not do that, depending on the Most Significant Bit (this is done for not losing the sign information located mainly on MSB).\\
\begin{center}
\begin{table}[ht!]
\centering
\begin{tabular}{|l|l|l|}
\hline
Operation        & \ms{01100010} & \ms{10100010} \\ \hline
$<<3$            & \ms{00010000} & \ms{00010000} \\ \hline
Log. $>>2$       & \ms{00011000} & \ms{00101000} \\ \hline
Arithmetic $>>2$ & \ms{00011000} & \ms{11101000} \\ \hline
\end{tabular}
\end{table}
\end{center}
\subsection{Encoding integers}
We define two functions to interpret the meaning of a binary variable $x$ depending on the type (whether it is \ms{unsigned} or not).\\
For \ms{unsigned}:\\
\begin{equation*}
B2U_{w}(x) = \sum_{i=0}^{w-1} x_i 2^i 
\end{equation*}
For signed \ms{int} (will be called as Two's Complement, or ``T"):\\
\begin{equation*}
B2T_{w}(x) = -x_{w-1}2^{w-1} + \sum_{i=0}^{w-2} x_i 2^i 
\end{equation*}



\end{document}